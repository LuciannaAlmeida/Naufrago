\documentclass[12pt,a4paper]{article}
\usepackage[brazil]{babel}
\usepackage[utf8]{inputenc}
\usepackage[T1]{fontenc}

\usepackage{syntonly}

\title{Náufragos\\Relatório}
\author{Lucas R. Colucci \and Lucianna T. Almeida \and Cindy de Albuquerque}

\pagestyle{headings}

%\syntaxonly

\usepackage[pdftex,bookmarks,colorlinks]{hyperref}

\begin{document}
\maketitle

\newpage

\tableofcontents
\newpage

\section{Introdução}
Neste documento serão apresentados: um manual do usuário (contendo os pré-requisitos do programa, como instalá-lo,
 como jogar e etc.), uma explicação de como foi implementado o projeto
 (contendo os módulos utilizados e respectivas explicações) e uma conclusão sobre a programação do programa.
Abaixo segue a história do jogo:\\

\emph{Estamos nos primeiros anos do século XX. O imponente transatlântico RMS Positronic deixa a Europa rumo ao Novo Mundo. Tudo corria bem em sua viagem, até que o navio esbarrou em um iceberg (sim, clichê básico de tragédias marítimas). Todavia, para a felicidade dos passageiros, o navio seguia rigorosamente os protocolos de segurança e havia coletes salva-vidas para todos. Outra sorte é que era verão na época do naufrágio, dessa forma tampouco as vítimas tiveram problemas com hipotermia.}

\emph{Felizmente para os passageiros, quis o destino que próximo ao naufrágio passasse o modesto navio cargueiro Asimov. O cargueiro, que contava com dois botes, prontamente colocou-os para o resgate dos passageiros em alto-mar. }
 
\section{Manual do Usuário}

Nesta seção explicaremos:
\begin{itemize}
\item os pré-requisitos para rodar o jogo
\item como instalar o jogo
\item os possíveis parâmetros a serem passados na linha de comando
\item como jogar
\end{itemize}
 

\subsection{Pré-requisitos}
Para poder rodar o programa em sua máquina, é necessária apenas a instalação da biblioteca \emph{Allegro} no seu computador.
Se você usa Ubuntu o jeito mais fácil de instalá-la é digitar no terminal a seguinte linha de comando:

\verb|sudo apt-get install liballegro-dev|

\subsection{Instalação}
\subsection{Parâmetros}
Existem alguns parâmetros que podem ser passados na linha de comando na hora de rodar o jogo, que são:
\begin{itemize}
\item -h        imprime o menu \emph{help}
\item -v<valor> seta o valor da \emph{velocidade média}
\item -s<valor> seta o valor da \emph{semente} usada no \verb|rand()|
\item -f<valor> seta o valor da \emph{frequência de geração de passageiros}
\end{itemize}

\subsection{Jogo}
Nessa subseção, serão esclarecidas as regras e os comandos do jogo.

\subsubsection{Regras}
O jogo começa com os dois botes (triângulos isóceles) lado a lado na parte inferior da tela.
Cada bote começa com 3 vidas extras.
O objetivo do jogo é salvar os passageiros (círculos verdes) naufragados. Para isso deve-se passar com o bote por cima dos passageiros, e despejá-los no \emph{Asimov} (retangulo preto), ancorando ao seu lado.

No entanto, existem algumas complicações:
\begin{itemize}
\item Não é possível carregar mais do que 10 passageiros de uma só vez
\item Existe uma área em volta do \emph{Asimov} para despejar os passageiros
\item Perde-se uma vida sempre que o bote encosta em um coral (quartos de circulos laranjas) 
\end{itemize}

Ganha o jogo quem, ao final, tiver o maior número de pontos.
Estes são distribuídos da seguinte maneira:
\begin{itemize}
\item Cada vez que um bote pega um passageiro, ganha-se 10 pontos
\item Cada vez que um bote despeja os passageiros no \emph{Asimov}, ganha-se 20 pontos para cada passageiro despejado
\item Ganha-se uma vida a cada 500 pontos acumulados
\end{itemize}

E, finalmente, o jogo acaba quando ambos os botes afundam, ao final de um minuto de jogo, ou quando o jogador aperta ESC.

\subsubsection{Comandos}
Os comandos utilizados para jogar são:\\

\textbf{ BOTE 1 - ESQUERDA }
\begin{itemize}
\item Acelera             - Botão W
\item Freia               - Botão S
\item Vira para esquerda  - Botão A
\item Vira para a direita - Botão D
\item Joga ancora         - Botão Q
\end{itemize}


\textbf{BOTE 2 - DIREITA}
\begin{itemize}
\item Acelera             - Direcional para cima
\item Freia               - Direcional para baixo
\item Vira para esquerda  - Direcional para esquerda
\item Vira para a direita - Direcional para direita
\item Joga ancora         - Ctrl do lado direito
\end{itemize}

\section{Implementação}
Nesta seção falaremos sobre a implementação do programa, ou seja, sua estruturação, os testes, e os interpretadores.

\subsection{Estruturação do código}
O programa foi separado em diferentes módulos, para facilitar tanto a atualização do jogo em cada estágio quanto a sua organização.
Os módulos feitos foram:\\

\textbf{asimov.c \& asimov.h}

Contém as funções que envolvem o asimov.\\

\textbf{bote.c \& bote.h}

Contém a estrutura do bote, e o vetor \emph{botes}, para facilitar o manuseio das estruturas.

Além disso, contém as funções que envolvem os botes, como funções auxiliares para sua movimentação.\\

\textbf{colisoes.c \& colisoes.h}

Arquivo que trata todos os tipos de colisões que acontecem durante jogo.\\

\textbf{coral.c \& coral.h}

Contém a estrutura do coral, e um vetor de corais.

Além disso possui funções auxiliares para descrever a localização do coral.\\

\textbf{estado\_inicial.c \& estado\_inicial.h}

Descreve o estado inicial do programa.\\

\textbf{imprime\_estado\_atual.c \& imprime\_estado\_atual.h}

Desenha na tela, com a biblioteca allegro, o estado atual do jogo.\\

\textbf{imprime\_informacoes.c \& imprime\_informacoes.h}

Imprime as informações do jogador na tela (nome, pontos, numero de passageiros a bordo), além de imprimir o fim de jogo.\\

\textbf{jogo.c \& jogo.h}

Onde sao chamadas todas as funções necessárias para rodar o programa.\\

\textbf{movimenta\_elementos.c \& movimenta\_elementos.c.h}

Contém funções de movimentação de todos os elementos.\\

\textbf{naufrago.c \& naufrago.h}

Contém a estrutura do passageiro e as funções a seu respeito.\\

\textbf{teclas.c \& teclas.h}

Contém o controle do jogador pelas teclas, definindo para que serve cada uma.\\

\textbf{visualizacao\_grafica.c \& visualizacao\_grafica.h}

Contém todas as funções necessárias para fazer a impressão do jogo na tela.\\

\textbf{minunit.c}

Contém as funções de teste.\\

\textbf{main.c}

Determina alguns valores definidos por parâmetros da linha de comando e inicia o jogo.\\


\subsection{Testes}
\subsection{Interpretadores}

\section{Conclusões}

O projeto nos proporcionou um lado mais "divertido"  da programação pois, ao final de cada etapa, tinhamos um resultado
concreto de nosso trabalho.

Como foi um trabalho extenso, e foi dado em grupo, tivemos idéia de como seria trabalhar com programação.

Além disso conseguimos aplicar quase toda, se nao toda, a matéria dada na disciplina de Laboratório de Programação I,
o que facilitou o entendimento dos topicos dados em sala de aula.



\end{document}
